\section{Introduction}
Graphs of vertices and edges are a mathematical abstraction that has been used commonly to refer to people or objects and their connections.
Most recently, graphs have expanded to depicting structures that we see in technology.
Especially worth noting lately are graphs being used to depict social media networks, the internet, and other complex social structures.
In particular, it is unique that these new graphs can evolve over time.
Due to their ability to hang in data over the internet, these graphs can evolve to many different states.
We can model these changes through an evolving graph, which is a collection of states of the graph at different points in time.

This new type of graph leads to the question: can we predict what the graph will look like in the future?
This can be used to help predict and recommend choices of friends on social networks or predict connections and popular sites
that may arise on the internet.
One method of predicting the behaviour of evolving graphs has been proposed by~\cite{grindrod2009}, where they propose a model
for evolving graphs that can be solved as an inverse problem.
Using the techniques I've learned in CS 6958, I explore solving this inverse problem. In particular, I have the following goals:

\begin{itemize}
	\item To solve the inverse problem faster using methods from class
	\item To solve the inverse problem more accurately
	\item To test these solutions on additional real data (which is made possible by speeding up these solutions)
\end{itemize}

As the current solution takes $O(n^3)$, solving the inverse problem on large datasets is not plausible.
By speeding up the solution, I hope to apply this inverse problems to datasets that are larger than the
~100 vertex graphs tested in~\cite{grindrod2009}.